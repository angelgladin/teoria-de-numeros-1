%%%
 %
 % Copyright (C) 2019 Ángel Iván Gladín García
 %
 % This program is free software: you can redistribute it and/or modify
 % it under the terms of the GNU General Public License as published by
 % the Free Software Foundation, either version 3 of the License, or
 % (at your option) any later version.
 %
 % This program is distributed in the hope that it will be useful,
 % but WITHOUT ANY WARRANTY; without even the implied warranty of
 % MERCHANTABILITY or FITNESS FOR A PARTICULAR PURPOSE.  See the
 % GNU General Public License for more details.
 %
 % You should have received a copy of the GNU General Public License
 % along with this program.  If not, see <http://www.gnu.org/licenses/>.
%%%

%%%%%%%%%%%%%%%%%%%%%%%%%%%%%%%%%%%%%%%%%%%%%%%%%%%%%%%%%%%%%%%%%%%%%%%%%%%%%%%%%%%%%%%%%
\documentclass[11pt,letterpaper]{article}
\usepackage[utf8]{inputenc}
\usepackage[spanish]{babel}
\usepackage[margin=1in]{geometry}
 
\usepackage{listings}
\usepackage{color}
\usepackage{graphicx}
\usepackage{enumerate}
\usepackage{enumitem}

\usepackage{longtable}
\usepackage{hyperref}
\usepackage{commath}

\usepackage{bbm}
\usepackage{dsfont}
\usepackage{mathrsfs}
\usepackage{amsmath,amsthm,amssymb}
\usepackage{mathtools}
%%%%%%%%%%%%%%%%%%%%%%%%%%%%%%%%%%%%%%%%%%%%%%%%%%%%%%%%%%%%%%%%%%%%%%%%%%%%%%%%%%%%%%%%%

\theoremstyle{definition}\newtheorem{p}{Ejercicio}
\theoremstyle{definition}\newtheorem{pp}[p]{$(*)$Ejercicio}
\numberwithin{p}{section}

\renewcommand{\theenumi}{\alph{enumi}}

%%%%%%%%%%%%%%%%%%%%%%%%%%%%%%%%%%%%%%%%%%%%%%%%%%%%%%%%%%%%%%%%%%%%%%%%%%%%%%%%%%%%%%%%%
\newcommand{\Z}{\mathbb{Z}}
\newcommand{\N}{\mathbb{N}}
\newcommand{\Q}{\mathbb{Q}}
\newcommand{\R}{\mathbb{R}}
\newcommand{\Oh}{\mathcal{O}} %% Notacion "O"
\newcommand{\ent}{\Longrightarrow}
\newcommand{\lra}{\longrightarrow}
\newcommand{\ra}{\rightarrow}
\newcommand{\sii}{\Longleftrightarrow}
\newcommand{\clase}[1]{\overline{#1}}  %% barrita sobre letras
\newcommand{\ord}{\text{ord}}
\newcommand{\leg}[2]{\left( \frac{#1}{#2}\right)} %% Simbolo de Legendre
\newcommand{\sol}{\textbf{\underline{Solución}: }} %% Solución

%%%%%%%%%%%%%%%%%%%%%%%%%%%%%%%%%%%%%%%%%%%%%%%%%%%%%%%%%%%%%%%%%%%%%%%%%%%%%%%%%%%%%%%%%

\begin{document}

%%%%%%%%%%%%%%%%%%%%%%%%%%%%%%%%%%%%%%%%%%%%%%%%%%%%%%%%%%%%%%%%%%%%%%%%%%%%%%%%%%%%%%%%%
\title{
    \vspace{-2cm}
        Universidad Nacional Autónoma de México\\
        Facultad de Ciencias\\
        Teoría de los Números I\\
    \vspace{.5cm}
    \large
        \textbf{Tarea 1}\\        
}
\author{
    Ángel Iván Gladín García\\
    No. cuenta: 313112470\\
    \texttt{angelgladin@ciencias.unam.mx}
}
\date{25 de Febrero 2019}
\maketitle
%%%%%%%%%%%%%%%%%%%%%%%%%%%%%%%%%%%%%%%%%%%%%%%%%%%%%%%%%%%%%%%%%%%%%%%%%%%%%%%%%%%%%%%%%

%%%%%%%%%%%%%%%%%%%%%%%%%%%%%%%%%%%%%%%%%%%%%%%%%%%%%%%%%%%%%%%%%%%%%%%%%%%%%%%%%%%%%%%%%
\newtheorem{theorem}{Teorema}
\newtheorem{example}{Ejemplo}
\newtheorem{corollary}{Corolario}
\newtheorem{lemma}{Lemma}
\newtheorem{definition}{Definición}
\newtheorem{prop}{Proposición}
%%%%%%%%%%%%%%%%%%%%%%%%%%%%%%%%%%%%%%%%%%%%%%%%%%%%%%%%%%%%%%%%%%%%%%%%%%%%%%%%%%%%%%%%%


%%%%%%%%%%%%%%%%%%%%%%%%%%%%%%%%%%%%%%%%%%%%%%%%%%%%%%%%%%%%%%%%%%%%%%%%%%%%%%%%%%%%%%%%%

\section{Divisibilidad}

%%%%%
\begin{p} %% Ejercicio 1.1
  Definimos la siguiente relaci\'on:
  \[
    a\preccurlyeq b\quad\stackrel{\text{def}}{\sii}\quad a\mid b.
  \]
  Prueba que $\preccurlyeq$ es un orden parcial sobre $\Z^+=\{1,2,\ldots\}$, es decir que $(\Z^+,\preccurlyeq)$.
  Explica porqu\'e no es un orden parcial sobre $\Z$.\\
  
  \sol Antes de empezar procederemos dando unas definiciones.
  
  \begin{definition}
  Un \textbf{orden parcial} es una relación binaria $R$ sobre un conjunto $X$ que es reflexiva, 
  antisimétrica, y transitiva, es decir, para cualesquiera $a$, $b$, y $c$ en $X$ se tiene que:
  \begin{enumerate}
      \item $aRa$ (reflexividad).
      \item Si $aRb$ y $bRa$, entonces $a=b$ (antisimetría).
      \item Si $aRb$ y $bRc$, entonces $aRc$ (transitividad).
  \end{enumerate}
  \end{definition} 
  
  \begin{definition}
  Sean $a, b \in \Z$, decimos que $a$ \textbf{divide} a $b$ si existe un entero $k \in \Z$ tal 
  que $b=ak$.
  \[
    a \mid b \quad\stackrel{\text{def}}{\sii} \quad  \exists k \in \Z \backepsilon b=ak
  \]
  \end{definition}
  
  Por demostrar que  $(\Z^+,\preccurlyeq)$. Para esto hay que probar las tres propiedades descritas
  anteriormente.
  \begin{itemize}
      \item \underline{Reflexividad}. Sea $a \in \Z$, por demostrar que $a \preccurlyeq a$.
      Por definición se tiene que $\exists k \in \Z \backepsilon a=ak$ y $k$ debe ser una unidad en 
      $\Z$ por lo que $a1=a$. Por tanto $a \preccurlyeq a$.
      
      \item \underline{Antisimetría}. Sean $a, b \in \Z$. Si $a \preccurlyeq b$ y $b \preccurlyeq a$ 
      por demostrar que $a=b$.
      Por definición de $\preccurlyeq$ tenemos que $\exists p \in \Z \backepsilon b=ap$ y 
      $\exists q \in \Z \backepsilon a=bq$ entonces sustituyendo $a$ tenemos que $b=(bq)p$, asociando 
      $b=b(qp)$, entonces $qp=1$. Por tanto $a=b$.

    \item \underline{Transitividad}. Sean $a, b, c \in \Z$. Si $a \preccurlyeq b$ y $b \preccurlyeq c$ 
      por demostrar que $a \preccurlyeq c$.
      Por definición de $\preccurlyeq$ tenemos que $\exists p \in \Z \backepsilon b=ap$ y 
      $\exists q \in \Z \backepsilon c=bq$, sustituyendo $b$ tenemos que $c=(ap)q$, asociando
      $c=a(pq)$. Por tanto $a \preccurlyeq b$.      
  \end{itemize}
    
    Ahora bien $(\Z, \preccurlyeq)$ no es un orden parcial porque no cumple la antisimetría y 
    se dará un contraejemplo.
    Sean $a, b \in \Z$. Si $a \preccurlyeq b$ y $b \preccurlyeq a$. Por definición de $\preccurlyeq$ 
    tenemos que $\exists p \in \Z \backepsilon b=ap$ y $\exists q \in \Z \backepsilon b=aq$ pero si 
    tomamos a $a=1$ y $b=-1$ (s.p.d.g), no se cumple $\preccurlyeq$ porque $a \neq b$.
\end{p}
%%%%%

%%%%%
\begin{p} Sobre las unidades de un anillo:
  \begin{enumerate}
  \item Sea $A$ un anillo conmutativo con 1 y
    $U(A)=\{u\in A\mid \exists v\in A\;\text{tal que}\; uv=1\}$ su conjunto de unidades. Definimos la
    siguiente relaci\'on:
    \[
      a\sim b \quad\stackrel{\text{def}}{\sii}\quad \exists u\in U(A)\;\;\text{tal que}\;\; a=ub
    \]
    Prueba que $\sim$ es una relaci\'on de equivalencia. Si $a\sim b$, decimos que $a$ y $b$ son
    \emph{asociados}. ?`Qu\'e conjunto es el espacio cociente $\Z/_{\sim}$?\\
    \sol TODO
    
  \item Sea $p\in\Z$ un n\'umero primo. Prueba que $\Z_{(p)}:=\{\frac{a}{b}\in\Q\mid p\not\,\mid b\}$
    es un anillo con las operaciones usuales de $\Q$ y describe el conjunto $U(\Z_{(p)})$.\\
  \sol \textbf{Quiero que este sea mi ejercicio gratis.}
    
  \item Prueba que cuales quiera dos elementos primos de $\Z_{(p)}$ son asociados (un elemento
    $q$ en cualquier anillo conmutativo con 1 es primo si no es una unidad y adem\'as cumple
    que $q\mid ab \ent q\mid a$ \'o $q\mid b$).\\
  \sol \textbf{Quiero que este sea mi ejercicio gratis.}
  
  \item Prueba que si $\frac{a}{b}\in\Z_{(p)}$ no es una unidad, entonces
    $\frac{a}{b}+1\in U(\Z_{(p)})$. Explica porque la prueba de Euclides de la infinitud de los
    n\'umeros primos falla para $\Z_{(p)}$.\\
  \sol \textbf{Quiero que este sea mi ejercicio gratis.}
  
  \end{enumerate}
  
\end{p}
%%%%%

%%%%%
%
\begin{p} Sean $a,b,c\in\Z$. Prueba las siguientes propiedades:
  \begin{enumerate}
  \item $a\mid b \ent ac\mid bc$ para toda $c\in\Z$ y si $c\neq 0$, entonces $ac\mid bc \ent a\mid b$. \\
  \sol Como por hipótesis y aplicando la definición de divisibilidad tenemos que $\exists p, q$ tales que 
  si $ap=b$ entonces $acq=bc$. Entonces basta con dividir $acq=bc$ entres $c$ ya que es un factor común 
  teniendo $\frac{acq}{c}=\frac{bc}{c}$, teniendo entonces $aq=b$. Por tanto $a \mid b$
  
  \item Si $a\mid a'$ y $b\mid b'$, entonces $a b\mid a'b'$. \\
  \sol Basta escribir a $a \mid a'$ y $b\mid b'$, usando la definición de divisibilidad tenemos entonces 
  $p, q \in \Z$ tal que $ap=a'$ y $bq=b'$. Multiplicando ambas igualdades tenemos que $apbq=a'b'$ y asociando 
  $ab(pq)=a'b'$. Por tanto $a b \mid a'b'$.
  
  \item Si $a\mid c$, $b\mid c$ y $(a,b)=1$, entonces $ab\mid c$. Muestra un contraejemplo de esta 
  propiedad si $(a,b)>1$. \\
  \sol Tomando $a=2$, $b=4$ y $c=4$. Porque $(a,b) = (2, 4) = 2$ y tenemos que  $2 \mid 4$ y $4 \mid 4$, 
  pero no se cumple $ab\mid c$ ya que $8 \nmid 4$.
  
  \item $(a+n,n)\mid n$ para toda $n\in\Z$.\\
  \sol Sabemos por un teorema, que el máximo común divisor de dos números cualesquiera puede ser expresado 
  como la mínima combinación lineal. Sabiendo éso, podemos expresar a $(a+n, n) = d$ como $p(a+n) + qn = d$ 
  con $p, q \in \Z$.  Ahora bien, por el inciso h) sabemos que si $(a,b)=d$, entonces $(\frac{a}{d}, \frac{b}{d}) = 1$. 
  Aplicando el resultado previo, tenemos que $(\frac{a+n}{d}, \frac{n}{d}) = 1$, reescribiendo como combinación 
  lineal se tiene que $r\frac{a+n}{d} + s\frac{n}{d} = 1$ para algún $r, s \in \Z$, multiplicando ambos lados por $n$ 
  se tiene $nr\frac{a+n}{d} + ns\frac{n}{d} = n$ y factorizando $n$ tenemos $n(r\frac{a+n}{d} + s\frac{n}{d}) = n$,
  lo que implica que $(r\frac{a+n}{d} + s\frac{n}{d}) = 1$. Por tanto, aplicando la definición de divisibilidad 
  se tiene que $(a+n,n)\mid n$.
  
  \item Si $(a,b)=1$ entonces $(a+b,a-b)=1$ \'o 2.\\
  \sol Si $(a,b)=d$ por definición se tiene que $d \mid a$ y $d \mid b$. Entonces regresando  a la expresión a 
  probar se tiene que $d \mid a+b, a-b$, si tomamos la suma y diferencia de ambos términos tenemos que 
  $d \mid (a+b)+(a-b)=2a$ y $d \mid (a+b)(a-b)=2b$, teniedo entonces que $d \mid (2a,2b)=2(a,b)=2$. 
  Ergo $d=1$ o $d=2$.
  
  \item $(a+tb,b)=(a,b)$ para toda $t\in\Z$.\\
  \sol Sabemos que podemos expresar a $(a,b)=d$ como la menor combinación lineal positiva. Entonces 
  expresamos a  $(a+tb,b)=d$ como dicha combinación teniendo entonces $(a+tb)m + bn = d$ con $m,n \in \Z$, 
  expandiendo el producto y asociando tenemos $am + (tm+n)b = d$ y por otro lado tenemos que 
  $ap' + bq' = d$. Por tanto, podemos expresar a ambos $(a+tb,b)$ y $(a,b)$ como la menor combinación lineal 
  positiva y ambos tienen el mismo máximo común divisor. Ergo $(a+tb,b)=(a,b)$ $\forall$ $t \in\Z$.
  
  \item Si $a'\mid a$, $b'\mid b$ y $(a,b)=1$ entonces $(a',b')=1$. En palabras esto es: si $a$ y $b$ son 
  primos relativos, entonces sus divisores son primos relativos entre ellos.\\
  \sol Por demostrar que $(a',b')=1$. Una de las hipótesis dice que $(a,b)=1$, reescribiendo el m.c.d. como
  $ax+by=1$ para algunos $x,y \in \Z$. Se sigue que:
  \begin{align*}
  ax+by=1 && \text{(Mínima combinación lineal de $(a,b)$)}\\
  a'p=a \quad \land \quad b'q=b && \text{(Por hipótesis y aplicando definición de divisibilidad)}\\
  (a'p)x + (b'q)y = 1 && \text{(Sitituyendo)}\\
  a'(px) + b'(qy) = 1 && \text{(Asociando)}\\
  \end{align*}
  Ergo si $a$ y $b$ son primos relativos, entonces sus divisores son primos relativos entre ellos.
  
  \item Si $(a,b)=d$ entonces $\left( \frac{a}{d}, \frac{b}{d} \right)=1$. \\
  \sol Sea $\left( \frac{a}{d}, \frac{b}{d} \right)=d'$. Por demostrar que $d'=1$. Como $d'$ es un factor 
  común de $\frac{a}{d}$ y de $\frac{b}{d}$, entonces $\exists l, m$ tales que $\frac{a}{d} = ld'$ y 
  $\frac{b}{d} = md'$. Entonces $a = ldd'$ y $b = m dd'$, entonces $dd'$ es un factor común de $a$ y $b$. 
  Entonces, por definición de máximo común divisor tenemos que $dd' \leq d$, entonces $d' = 1$. 
  Por tanto $d'$ es un entero positivo tal que $d'=1$. Ergo si  $(a, b) = d$ entonces  
  $\left( \frac{a}{d}, \frac{b}{d} \right)=1$.
  
  \item Si $(a,b)=1=(a,c)$, entonces $(a,bc)=1$.\\
  \sol Por demostrar que $(a,bc)=1$. Se tiene por hipótesis que $(a,b)=1=(a,c)$ lo cual se puede expresar como
  $ax+by=1$ y $ap+cq=1$ para algunos $x,y,p,q \in \Z$. Se sigue entonces:
  \begin{align*}
  (ax+by)(ap+cq)=1 && \text{(Multiplicando ambas ecuaciones)}\\
  axap+axqp+byap+bycq = 1 && \text{(Aplicando producto)}\\
  a(xap+ xqp + byp) + bc(yq) = 1 && \text{(Factorizando y asociando)}\\
  \end{align*}
  Por tanto $(a,bc)=1$.
  
  \item  Sea $a_0,a_1,a_2,\ldots$ la sucesi\'on de Fibonacci 1,1,2,3,5.$\ldots$ definida recursivamente como 
  $a_{n+1}:=a_n+a_{n-1}$ donde $a_0=1=a_1$. Prueba que $(a_n,a_{n+1})=1$ para toda $n$.\\
  \sol Prueba por inducción. Sea $n \in \N$, demostrar que la propiedad $P(n)$ se cumple $\forall n \in \N$\\
  \textbf{Caso base:} para $n=2$, $P(2) = a_2 = a_1 + a_0=2$. Se cumple que $(a_1, a_2) = (2,1) = 1$.\\
  \textbf{Hipótesis de inducción:} Suponer que $k \in \N$ con $k>1$, entonces se cumple $P(k)$ tal que
   $(a_k, a_{k+1})= 1$ para toda $k \in \Z$.\\
   \textbf{Paso inductivo:} Probar que se cumple $P(n+1)$.
   \begin{align*}
   (a_{k+1}, a_{k+2})
       &= (a_{k+1}, a_{k+1} +a_k) && \text{(Definición de Fibonacci, $a_{k+2}:=a_{k+1}+a_{k}$)}\\
       &= (a_{k+1} +a_k, a_{k+1}) && \text{(Conmutanto)}\\
       &= (a_{k+1} , a_{k}) && \text{(Usando que $(a,b) = (a+b, a)$)}\\
       &= 1 && \text{(Por hipótesis de inducción)}
   \end{align*}
   Por tanto $(a_n,a_{n+1})=1$ $\forall n$.
   
  
  \end{enumerate}
\end{p}
%
%%%%%


\begin{p}
  Un \emph{m\'inimo com\'un m\'ultiplo} de dos enteros $a,b\in\Z$ se define como un entero $m> 0$
  que cumple las siguientes dos propiedades:
  \begin{enumerate}
  \item[$(\bullet)$] $a\mid m$ y $b\mid m$.
  \item[$(\bullet\bullet)$] Si $m'\in\Z$ es tal que $a\mid m'$ y $b\mid m'$, entonces $m\mid m'$.
  \end{enumerate}
  Fija $a,b,c\in\Z$. Prueba las siguientes propiedades del m\'inimo com\'un m\'ultiplo (mcm):
  \begin{enumerate}
  \item Prueba que el mcm de $a,b$ es \'unico; gracias a esto lo denotamos por $[a,b]$.\\
  \sol Prueba por unicidad y existencia.\\
  Empecemos por suponer $n,m \in \Z$ que satisfacen $(\bullet)$ y 
  $(\bullet \bullet)$. Como $m$ es un m.c.d. de $a$ y $b$ y como $n$ satisface $(\bullet \bullet)$ 
  entonces $n \mid m$. Sin perdida de generalidad, intercambiamos las variables y tenemos que
  $m \mid n$. Lo que implica que si $n \mid m$ y $m \mid n$, entonces $n=m$ y con esto 
  hemos provado la unicidad.\\
  Para provar la existencia, denotaremos a $S = {x \in \N : a \mid x \land b \mid x}$. Por el 
  principio del buen orden se tiene que $S$ tiene un elemento mínimo $m$. Por tanto tenemos que 
  $a \mid m$ y $b \mid m$, con ésto hemos provado $(\bullet)$. Para probar $(\bullet \bullet)$ 
  denotaremos a $x \in \Z$ tal que $a \mid x \land b \mid x$. Por el algorítmo de la división 
  tenemos que $\exists! q, r \in \Z$ tal que $x = mq+r, \quad 0 < r \leq m$. Dado que $a \mid x$
  y $a \mid x$ entonces $x = aq_1$ para algún $q_1 \in \Z$. Como $a \mid m$ entonces por definición 
  $m = aq_2$ para alguna $q_2 \in \Z$. Se sigue que $aq_1 = aq_2+r$ y reescribiendo $a(q_1 - q_2)= r$. 
  Lo que implica que $a \mid r$ (por definición). Si $r>0$ entonces $r \in S$ y contradice la 
  definición de $m$. Ergo $r = 0$. Para $b$ la prueba es totalmente análoga.
  
  \item $[ab,ac]=a[b,c]$\\
  \sol En el inciso $d)$ se demostró qué $[a,b]=\frac{ab}{(a,b)}$, por consiguiente podemos expresar a
  $[ab,ac]$ como $[ab,ac]=\frac{abac}{(ab,ac)}$ de lo que se sigue que:
  \begin{align*}
  [ab,ac]
      &= \frac{abac}{(ab,ac)} && \text{(Por prueba del inciso $d)$)}\\
      &= \frac{abac}{abx+acy} && \text{(Mínima combinación lineal de ab, cy)}\\
      &= \frac{abac}{a(bx + cy)} && \text{(Factorizando)}\\
      &= \frac{abc}{bx+cy} && \text{(Reduciendo factor común)}\\
      &= a \frac{bc}{(b,c)} && \text{(Reescribiendo como m.c.d a $bx+cy$)}\\
      &= a [b, c] && \text{(Por inciso $d)$)}
  \end{align*}
  Ergo $[ab,ac]=a[b,c]$.
  
  \item $(a,b)=[a,b] \ent a=b$\\
  \sol Tenemos por el inciso $d)$ que $[a,b]=\frac{ab}{(a,b)}$. Sabiendo eso, rescribiremos al m.c.m. como 
  $[a,b] = \frac{ab}{(a,b)}$ pero por hipótesis se tiene que $(a,b)=[a,b]$, sustituyendo $[a,b]$ se tiene que 
  $(a,b) = \frac{ab}{(a,b)}$, despejando se sigue que $(a,b)^2 = ab \iff a=b$ porque $ab$ tiene que 
  ser cuadrado.
  
  \item $ab=(a,b)[a,b]$\\
  \sol Antes de probarlo, enunciaremos los siguiente Lemas. Sean $a$ y $b$ dos enteros positivos  
  expresaremos al m.c.m y m.c.d. con la siguiente descomposición canónica.
  $$a=p_{1}^{a_1}p_{2}^{a_2} \cdots p_{n}^{a_n} \quad \land  \quad b=p_{1}^{b_1}p_{2}^{b_2} \cdots p_{n}^{b_n} \backepsilon a_i,b_i \geq 0$$
  (Asumimos que ambas descomposiciones contienen exactamente las mismas bases primas $p_i$). 
  Entonces podemos expresar al m.c.m y m.c.d como .
  $$[a,b] =p_{1}^{max(a_1,b_1)}p_{2}^{max(a_2,b_2)} \cdots p_{n}^{max(a_n,b_n)}$$
  $$(a,b) =p_{1}^{min(a_1,b_1)}p_{2}^{min(a_2,b_2)} \cdots p_{n}^{min(a_n,b_n)}$$
  Ahora bien, teniendo los lema anteriores procederemos con la demostración.\\
  Tenemos que $ab=(a,b)[a,b]$, reacomodando tenemos demostrar que $[a,b]=\frac{ab}{(a,b)}$.
  Sea $a=p_{1}^{a_1}p_{2}^{a_2} \cdots p_{n}^{a_n}$ y $b=p_{1}^{b_1}p_{2}^{b_2} \cdots p_{n}^{bn}$ 
  la descomposición canónica de $a$ y $b$. Entonces tenemos
  $$[a,b] = p_{1}^{max(a_1,b_1)}p_{2}^{max(a_2,b_2)} \cdots p_{n}^{max(a_n,b_n)}$$
  y
  $$(a,b) = p_{1}^{min(a_1,b_1)}p_{2}^{min(a_2,b_2)} \cdots p_{n}^{min(a_n,b_n)}$$
  Por consiguiente,
  \begin{align*}
  (a,b)[a,b]
    &= p_{1}^{min(a_1,b_1)}p_{2}^{min(a_2,b_2)} \cdots p_{n}^{min(a_n,b_n)} \cdot p_{1}^{max(a_1,b_1)}p_{2}^{max(a_2,b_2)} \cdots p_{n}^{max(a_n,b_n)} \\
    &= p_{1}^{min(a_1,b_1)+max(a_1,b_1)} \cdots p_{n}^{max(a_n,b_n)+max(a_n,b_n)} \\
    &= p_{1}^{a_1 + b_1}p_{2}^{a_2 + b_2} \cdots p_{n}^{a_n + b_n} \\
    &= (p_{1}^{a_1}p_{2}^{a_2} \cdots p_{n}^{a_n})(p_{1}^{b_1}p_{2}^{b_2} \cdots p_{n}^{b_n}) \\
    &= ab
  \end{align*}
  Ergo $[a,b]=\frac{ab}{(a,b)}  \iff ab=(a,b)[a,b]$  
  
  \item $(a+b,[a,b])=(a,b)$\\
  \sol Antes de proceder con la demostración enunciaremos un lema y un teorema que usaremos.
  \begin{lemma}
  Sean $a,b \in \Z$, Sea $d = (a,b)$ entonces $d = (a+b,a) = (a+b,b)$.\\
  \underline{Demostración:} 
  \begin{align*}
  (a,b) &= ax + by\\
    &= ax + by + (ay - ay)\\
    &= ay + by + ax - ay\\
    &= y(a+b) + a(x-y)\\
    &= (a+b)y + ap'
  \end{align*}
  Por tanto $(a,b) = (a+b,a)$
  \begin{align*}
  (a,b)
    &= ax + by\\
    &= ax + by + (bx - bx)\\
    &= ax + bx + by - bx\\
    &= x(a+b) + b(y-x)\\
    &= (a+b)x + bq'
  \end{align*}
  Por tanto $(a,b) = (a+b,b)$
  \end{lemma}

  \begin{theorem}
  El m.c.d. se distribuye sobre el m.c.m. Sean $a,b,c \in \Z$, se sigue que:
  $$(a,[b,c]) = [(a,b),(a,c)]$$
  \end{theorem}
  
  Sabiendo eso procederemos con la demostración:
  \begin{align*}
  (a+b, [a,b]) \\
    &= [(a+b,a), (a+b, b)] && \text{(Distributividad del m.c.d. sobre m.c.m)}\\
    &= \frac{(a,b)}{(a,b)} && \text{(Aplicando el lemma)}\\
    &= \frac{(a,b)(a,b)}{(a,b)} && \text{(Por inciso $d)$)}\\
    &= (a, b) && \text{(Simplificando)}
  \end{align*}
  Ergo $(a+b,[a,b])=(a,b)$.
  
  \end{enumerate}
\end{p}
%
\begin{p}
  Sean $a\in\Z$ y $d\in\Z^+$ fijos y considera el sistema de ecuaciones
  \[
    (\star)\left\{
      \begin{matrix}
        (x,y) = d\;\\
        xy =a
      \end{matrix}
    \right.
  \]
  Prueba que $(\star)$ tiene una soluci\'on $(x_0,y_0)\in\Z\times\Z$ si y solamente si $d^2\mid a$.\\
  \sol $\Longrightarrow	 )$ Por demostrar que $d^2 \mid a$. Tomando $(x_0,y_0)\in\Z\times\Z$ como una 
  solución particular de la forma $x_0=d$ y $y_0=\frac{a}{d}$, de esta forma se satisface en la segunda ec. que:
  $$d\frac{a}{d}=a \quad a=a$$
  Por definición de divisivilidad $\exists k \in \Z \backepsilon d^2k=a$ y escibiendo a como $(x,y)$ como 
  combinación lineal de la forma $xs+ yt = d$ p.a. $s,t \in \Z$ se sigue que:
  \begin{align*}
  a
      &= kd^2\\
      &= k(xs+yt)^2 && \text{(Sutituyendo $d=xs+yt$)}\\
      &= k(ds+\frac{a}{d}t)^2 && \text{(Reescribiendo $x=d$ y $y=\frac{a}{d}$)}\\
      &= k(d^2s^2 + 2ds\frac{a}{d}t + \frac{a^2}{d^2}t) && \text{(Expandiendo el binomio)}\\
      &= k(d^2s^2 + 2ds\frac{xy}{d}t + \frac{a^2}{d^2}t^2) && \text{(Reescribiendo $a=xy$)}\\
      &= k(d^2s^2 + 2ds\frac{dy}{d}t + \frac{a^2}{d^2}t^2) && \text{(Reescribiendo $x=y$)}\\
      &= k(d^2s^2 + d^2\frac{2syt}{d} + \frac{a^2}{d^2}t^2) && \text{(Asociando y conmutando)}\\
      &= k(d^2s^2 + d^2\frac{2syt}{d} + \frac{x^2y^2}{d^2}t^2) && \text{(Reescribiendo $a=xy$)}\\
      &= k(d^2s^2 + d^2\frac{2syt}{d} + d^2\frac{y^2t^2}{d^2}) && \text{(Reescribiendo $x=d$ y asociando)}\\
      &= kd^2(s^2 + \frac{2syt}{d} + \frac{y^2t^2}{d^2}) && \text{(Factorizando $d^2$)}\\
      &= m'd^2 && \text{Sea $m'=k(s^2 + \frac{2syt}{d} + \frac{y^2t^2}{d^2})$}
  \end{align*}
  Por definición de divisivilidad, ergo $d^2 \mid a$.\\
  $\Longleftarrow )$ Por demostrar que $(\star)$ tiene una solución $(x_0,y_0)\in\Z\times\Z$.
  Como $d^2 \mid a$ entonces $d^2p=a$ para alguna $p \in \Z$. Entonces tiene solucion
  \begin{align*}
  & \iff xy=d^2p\\
  & \iff d^2p=a\\
  & \iff x=d^2 \land y=\frac{a}{d^2}\\
  & \iff p=y=\frac{a}{d^2}
  \end{align*}
  Por tanto, $(\star)$ tiene una solución $(x_0,y_0)\in\Z\times\Z$.
\end{p}

%
\begin{p}
  Sea $D_n$ el conjunto de divisores positivos de $n$, ie. $D_n=\{d>0 : d\mid n\}$. Ahora sea
  $F:D_a\times D_b\ra D_{ab}$ la funci\'on definida por $F(d,d')=dd'$. Prueba que si $(a,b)=1$, entonces $F$ es
  una funci\'on bien definida y que es biyectiva. Si $(a,b)>1$, ?`deja de ser biyectiva la funci\'on? Explica tu
  respuesta.\\
  \sol TODO
\end{p}
%
\begin{p}
  Sean $n,m,x,y\in\Z$ fijos tales que $n=ax+by$ y $m=cx+dy$ para algunas $a,b,c,d\in\Z$. Si $ad-bc=\pm 1$, prueba
  que $(m,n)=(x,y)$.\\
  \sol Sea $d=(m,n)$, entonces podemos expresar a $d$ como $d=sn+tm$ p.a. $s,t \in \Z$, sea 
  $d'=(x,y)$. Como por hipótesis  dejamos fijos $n=ax+by$ y $m=cx+dy$, entonces 
  $d=s(ax+by)+t(cx+dy)=sax+sby+tcx+tdy=x(sa+tc)+y(sb+td)$. Por tanto se sigue que $d \mid d'$ 
  porque tanto $d$ y $d'$ se pueden expresar como combinaciones lineales en términos de $x$ y $y$.
  Una transformación lineal por $\begin{bsmallmatrix}a&b\\c&d\end{bsmallmatrix}$
  es invertible si $\Delta \neq 0$, pero como por hipótesis se tiene que $ad-bc=\pm 1$, entonces 
  podemos escribir a $x$ y $y$ como combinaciones lineales de $m$ y $n$, entonces se sigue que 
  $x=dm-bn$ y $y=-cm+an$. $\exists p, q \backepsilon px + qy = d'$.
  Se tiene que:
  \begin{align*}
  d'
    &= px + qy \\
    &= p(dm-bn) + q(-cm+an) \\
    &= m(pd-qc) + n(-qb+qa) \\
  \end{align*}
  Por tanto, se tiene que $d' \mid d$.\\
  Si $(m,n) \mid (x, y)$ y $(x, y) \mid (m,n)$, (como el máximo común divisor es positivo) 
  entonces $(m,n)=(x,y)$.
\end{p}
%

\begin{p}
  Fija tres enteros $a,b,c\in\Z$. Prueba que la ecuaci\'on $ax+by=c$ tiene soluci\'on si y solamente si $(a,b)\mid c$.
  Adem\'as, si $(x_0,y_0)$ es una soluci\'on ?`de qu\'e forma son el resto de las soluciones?\\
  \sol $\Rightarrow )$ TODO
  
\end{p}
%
\begin{p}
  Prueba que todo entero mayor que 6 se puede expresar como suma de dos enteros primos relativos.\\
  \sol Antes de proceder, se provará probar dos lemas que serán de gran utilidad para la siguiente demostración.
  \begin{lemma}
  Sean $a, b, d \in \Z$ si $d \mid a$ y $d \mid b$, entonces $d \mid a-b$.
  \begin{align*}
  dp = a \quad \land \quad dq=b && \text{(Definición de divisibilidad)}\\ 
  dp - dq = a -b && \text{(Restando la ec.2 a ec.1)}\\
  d(p-q) = a-b && \text{(Asociando)}
  \end{align*}
  Por tanto $d \mid a-b$.
  \end{lemma}
  \begin{lemma}
  \emph{Dos enteros consecutivos son primos relativos}. Sea $n \in \Z$, suponer que $(n, n+1)=p$, lo que
  por definición de divisivilidad se sigue que $p \mid n$ y $p \mid n+1$, por el lema anterior se sigue que
  $p \mid n - (n+1)$ lo que implica que  $p \mid 1$. Por definición de divisibilidad $pr=1$, pero esto ocurre
  si y solo si $r=1$. Por tanto $(n, n+1)=p=1$.
  \end{lemma}
  
  Teniendo ésos dos lemas, se continuará con la demostración. Sea $n>6$, se procederá a analizar por casos.
  \begin{itemize}
    \item Si $n$ es par. Entonces $n$ es de la forma $n=2k \backepsilon k \in \Z$ con $k \geq 4$
    resscribiendo a $n$ como $2k=k+k=(k+2)+(k-2)$ o $2k=k+k=(k+4)+(k-4)$, y  dados cualesquiera
    dos números impares $a$ y $b$ si su diferencia es $2$ ó $4$ se sigue que son primos relativos.
    \item Si $n$ es impar. Entonces $n$ es de la forma $n=2k+1 \backepsilon k \in \Z$ con $k \geq 3$ 
    resscribiendo a $n$ como $2k+1= k + (k +1)$ se sigue que $k$ y $k+1$ son ambos primos relativos
    por el lemma previamente citado.
    
  \end{itemize}
\end{p}
%
\begin{pp}
  Demuestra que para todo $a>1$ y exponentes $n,m>0$ se cumple que $(a^n-1,a^m-1)=a^{(n,m)}-1$.\\
  \sol TODO
  
\end{pp}
%
\begin{pp}
  Los n\'umeros arm\'onicos no son enteros. 
  \begin{enumerate}
  \item Sean $\frac{a}{b},\frac{c}{d}\in\Q$ fracciones irreducibles, es decir $(a,b)=1=(c,d)$. Prueba que
    \[
      \frac{a}{b}+\frac{c}{d}\in\Z \quad\ent\quad b=\pm d.
    \]
    \sol TODO
  \item Los n\'umeros arm\'onicos $H_n$ se definen como las sumas parciales de la serie arm\'onica, es decir
  \[
    H_n:=\sum_{k=1}^n\frac{1}{k}.
  \]
  Prueba que $H_n\not\in\Z$ para toda $n>1$.\\
  \sol TODO
  \end{enumerate}
\end{pp}


%%%%%%%%%%%%%%%%%%%%%%%%%%%%%%%%%%%%%%%%%%%%%%%%%%%%%%%%%%%%%%%%%%%%%%%%%%%%%%%%%%%%%%%%%


%%%%%%%%%%%%%%%%%%%%%%%%%%%%%%%%%%%%%%%%%%%%%%%%%%%%%%%%%%%%%%%%%%%%%%%%%%%%%%%%%%%%%%%%%
\begin{thebibliography}{}

\bibitem{} 
Thomas Koshy. 
\textit{Elementary Number Theory with Applications. 2nd Edition.}
Addison-Wesley, Reading, Massachusetts, 1993.
Academic Press. 
8th May 2007.

\bibitem{}
Apostol, Tom M. 
\textit{Introduction to Analytic Number Theory.} 
Springer-Verlag, New York, 1976.

\bibitem{}
Notas tomadas en clase del curso de Teoría de los Números I (2019-2).

\end{thebibliography}
%%%%%%%%%%%%%%%%%%%%%%%%%%%%%%%%%%%%%%%%%%%%%%%%%%%%%%%%%%%%%%%%%%%%%%%%%%%%%%%%%%%%%%%%%

\end{document}